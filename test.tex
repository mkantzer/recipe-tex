% !TEX program = xelatex

\documentclass[letterpaper]{article}
% \usepackage[utf8]{inputenc} % this is needed for umlauts
% \usepackage[ngerman]{babel} % this is needed for umlauts
% \usepackage[T1]{fontenc}    % this is needed for correct output of umlauts in pdf
\usepackage{siunitx}
\usepackage[nonumber]{cuisine}
\usepackage{microtype}
\usepackage{fontspec}

\setmainfont[Ligatures=TeX]{Comic Neue}
\setsansfont[Ligatures=TeX]{Comic Neue}


\usepackage[margin=0.125cm, layoutwidth=6in, layoutheight=4in]{geometry}
% Total | number of servings | step number | ingredient | quantity | units
\RecipeWidths{6in}{2in}{1em}{10em}{3em}{3em}
\begin{document}
% \renewcommand*{\recipefont}{\sffamily}
\begin{recipe}{Zabaglione alla Marsala}{4 Portions}{\fr12 hour}
  \freeform This is a well-known Italian recipe which is
  great for piling on the calories.
  \ingredient[6]{}{egg yolks}
  \Ing{\hspace{8em}literally anything}
  \ingredient[2]{oz}{granulated sugar}
  \ingredient[6--8]{tbsp}{Marsala (or sherry)}
  \ingredient[\fr14]{oz}{gelatine}
  \ingredient[2]{tbsp}{cold water}
  In the top of a double boiler, combine the egg yolks with sugar and
  Marsala, and whip the mixture over hot, but not boiling, water until it
  thickens. Stir in gelatine, softened in cold water and dissolved over hot
  water.
  \ingredient[3]{tbsp}{brandy}
  \ingredient[\fr38]{pt}{double cream}
  Put the pan in a bowl of ice and stir the
  zabaglione well until it is thick and free of bubbles.  When it is almost
  cold, fold in brandy and whipped cream and pour into individual moulds.
  \ingredient[3]{}{egg yolks}
  \ingredient[1]{oz}{granulated sugar}
  \ingredient[3--4]{tbsp}{Marsala (or sherry)}
  \ingredient[1\fr12]{tbsp}{brandy}
  To make the sauce, combine egg yolks and sugar in the top of a double
  saucepan.  Whisk mixture over hot, but not boiling, water until the
  sauce coats
  the back of a spoon.  Stir in the Marsala and brandy.
  \newstep
  Chill the zabaglione, unmould it, and serve with the sauce immediately.
  % \freeform\hrulefill
  \end{recipe}

\end{document}