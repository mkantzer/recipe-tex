% !TEX program = xelatex
\documentclass[a6paper,landscape]{article}
% \usepackage[utf8]{inputenc} % this is needed for umlauts
% \usepackage[ngerman]{babel} % this is needed for umlauts
% \usepackage[T1]{fontenc}    % this is needed for correct output of umlauts in pdf
\usepackage[nonumber]{cuisine}
\usepackage{fontspec}

\setmainfont[Ligatures=TeX]{Comic Neue}
\setsansfont[Ligatures=TeX]{Comic Neue}

% \usepackage[margin=0.0cm, layoutwidth=6in, layoutheight=4in]{geometry}
\usepackage[a6paper, landscape, margin=0.0cm]{geometry}
% Total | number of servings | step number | ingredient | quantity | units
\RecipeWidths{146mm}{10em}{1em}{8.5em}{3em}{2em}
\renewcommand*{\recipetitlefont}{\large\bfseries}
\renewcommand*{\recipestepnumberfont}{\bfseries}
\renewcommand*{\recipefreeformfont}{\slshape}

\begin{document}

\begin{recipe}{Coconut Chicken Curry        (NYT)}{1 Hour}{}
  \freeform Warm Curry. Serve with rice! 

  \ing[2\fr12]{lb}{chicken thigh}
  \ing[1]{TBSP}{paprika}
  \ing[\fr12]{tsp}{tumeric}
  \ing[2]{tsp}{kosher salt}
  Trim fat, cut into \fr12 - 1 in pieces. Add spices, mix well. Let sit at room temp while finishing prep
  \ing[\fr13]{cup}{canola oil}
  \ing[2]{}{yellow onions, diced}
  \ing[4]{(lol)}{garlic cloves}
  Large pot, heat oil. Sauté onions med-low 8-10min. Add garlic, 5 min
  \ing[13]{oz}{unsweetened coconut milk}
  \ing[1\fr12]{TBSP}{fish sauce}
  \ing[1\fr12]{cup}{water}
  Add chicken, stir to spice onion. Add coconut, almost boil, 4 min. add fish sauce, water, near-boil. 
  \ing[1]{tsp}{curry powder}
  \ing[\fr12]{tsp}{cayenne}
  Simmer, stirring, 30-45 min. Add spices, simmer briefly, remove from heat
  \newstep let sit 20 min to soak spice into chicken
  \Ing{Cooked Rice or Noodles}
  \ing[1]{cup}{cilantro}
  \ing[1]{}{lime, cut to wedges}
  Simmer right before serving on rice. Top w/ cilantro \& lime
  \end{recipe}

\end{document}