% !TEX program = xelatex
\documentclass[a6paper,landscape]{article}
% \usepackage[utf8]{inputenc} % this is needed for umlauts
% \usepackage[ngerman]{babel} % this is needed for umlauts
% \usepackage[T1]{fontenc}    % this is needed for correct output of umlauts in pdf
\usepackage[nonumber]{cuisine}
\usepackage{fontspec}

\setmainfont[Ligatures=TeX]{Comic Neue}
\setsansfont[Ligatures=TeX]{Comic Neue}

% \usepackage[margin=0.0cm, layoutwidth=6in, layoutheight=4in]{geometry}
\usepackage[a6paper, landscape, margin=0.0cm]{geometry}
% Total | number of servings | step number | ingredient | quantity | units
\RecipeWidths{146mm}{10em}{1em}{8.5em}{3em}{2em}
\renewcommand*{\recipetitlefont}{\large\bfseries}
\renewcommand*{\recipestepnumberfont}{\bfseries}
\renewcommand*{\recipefreeformfont}{\slshape}

\begin{document}

\begin{recipe}{Crispy Chicken Stir Fry}{20 minutes}{}
  \freeform Decent stir fry, that works with basically any veggie. 
  \Ing{2lb Chicken Breast, cut to 2in}
  \Ing{3 TBSP cornstarch }
  Toss cut up chicken in cornstarch.
  \Ing{2 TBSP sesame oil}
  In large skillet, heat oil till glistening. In small batches, add chicken and crisp, 8 min. Add oil if needed.
  \Ing{2 TBSP sesame oil}
  \Ing{\fr34 lb Green Beans, trimmed}
  Add oil and heat. Add green beans, until blistered, 5 min. Remove from pan.
  \Ing{3 minced garlic cloves}
  \Ing{1 TBSP minced ginger}
  \Ing{\fr14 cup soy sauce or tamari}
  \Ing{1 tsp chili paste}
  \Ing{\fr14 Sea Salt}
  Heat to medium low, cool off. Add Garlic, ginger, soy sauce, paste, salt. Stir till garlic is fragrant. 
  \newstep Add chicken and veggies back to pan, stir to coat.
  \Ing{Sesame Seeds}
  Serve with with sesame seeds

  \end{recipe}

\end{document}