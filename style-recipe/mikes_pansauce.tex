% !TEX program = xelatex
\documentclass[a6paper,landscape]{article}
% \usepackage[utf8]{inputenc} % this is needed for umlauts
% \usepackage[ngerman]{babel} % this is needed for umlauts
% \usepackage[T1]{fontenc}    % this is needed for correct output of umlauts in pdf
\usepackage[nonumber]{cuisine}
\usepackage{fontspec}

\setmainfont[Ligatures=TeX]{Comic Neue}
\setsansfont[Ligatures=TeX]{Comic Neue}

% \usepackage[margin=0.0cm, layoutwidth=6in, layoutheight=4in]{geometry}
\usepackage[a6paper, landscape, margin=0.0cm]{geometry}
% Total | number of servings | step number | ingredient | quantity | units
\RecipeWidths{146mm}{10em}{1em}{8.5em}{3em}{2em}
\renewcommand*{\recipetitlefont}{\large\bfseries}
\renewcommand*{\recipestepnumberfont}{\bfseries}
\renewcommand*{\recipefreeformfont}{\slshape}

\begin{document}


\begin{recipe}{Mike's Pan Sauce}{5 minutes}{}
  \freeform \\A decent pan sauce for burgers, chicken, whatever, using the bits from the cooked meat.\\
  \Ing{all leftover fond}
  \Ing{1 shallot, chopped}
  \Ing{drizzle of olive oil}
  \Ing{1 tsp rosemary, chopped}
  \Ing{mushrooms, chopped}
  Heat in pan @ medium. Add the rest. Cook, tossing, until softened.
  \Ing{1 TBSP jam/whatever}
  \Ing{\fr12 cup stock, or bullion cube}
  \Ing{\fr14 cup water}
  \Ing{5 tsp balsamic vinegar}
  Add the bits, stir to combine. Let simmer 2-3 minutes, until \textit{thick and saucy}
  \Ing{1 TBSP butter}
  \Ing{S\&P, to taste}
  Remove from heat. Add butter and season. Stir to melt.
  \newstep Serve over meat

  \end{recipe}

\end{document}