% !TEX program = xelatex
\documentclass[a6paper,landscape]{article}
% \usepackage[utf8]{inputenc} % this is needed for umlauts
% \usepackage[ngerman]{babel} % this is needed for umlauts
% \usepackage[T1]{fontenc}    % this is needed for correct output of umlauts in pdf
\usepackage[nonumber]{cuisine}
\usepackage{fontspec}

\setmainfont[Ligatures=TeX]{Comic Neue}
\setsansfont[Ligatures=TeX]{Comic Neue}

% \usepackage[margin=0.0cm, layoutwidth=6in, layoutheight=4in]{geometry}
\usepackage[a6paper, landscape, margin=0.0cm]{geometry}
% Total | number of servings | step number | ingredient | quantity | units
\RecipeWidths{146mm}{10em}{1em}{8.5em}{3em}{2em}
\renewcommand*{\recipetitlefont}{\large\bfseries}
\renewcommand*{\recipestepnumberfont}{\bfseries}
\renewcommand*{\recipefreeformfont}{\slshape}

\begin{document}

\begin{recipe}{Baked Sweet Hawaiian Chicken}{1h 20m}{}
  \freeform Warm Curry. Serve with rice! 

  \ing[3-4]{}{Chicken Breasts}
  \ing[]{taste}{Salt \& Pepper}
  Preheat oven to 325. Cut breasts into bite-sized pieces. Season.
  \ing[1\fr12]{cup}{Cornstarch}
  \ing[3]{}{Eggs, beaten}
  Put in different bowls. Dip chicken in starch, then eggs.
  \ing[\fr14]{cup}{Canola Oil}
  Heat in skillet, med-high. Brown Chicken, then move to 9x13 dish.
  \ing[1]{cup}{Pineapple Juice}
  \ing[\fr12]{cup}{Brown Sugar}
  \ing[\fr13]{cup}{Soy Sauce}
  \ing[1]{tsp}{Garlic, Minced}
  \ing[\fr12]{TBSP}{Cornstarch}
  \ing[1]{}{Red Pepper, Chopped}
  \ing[1]{can}{Pineapple Tidbits}
  Whisk together in a bowl. Pour over chicken.
  \newstep Bake, \textbf{1 hour}. Stir every \textbf{15 min}
  \newstep Serve with rice
  \end{recipe}

\end{document}