% !TEX program = xelatex
\documentclass[a6paper,landscape]{article}
% \usepackage[utf8]{inputenc} % this is needed for umlauts
% \usepackage[ngerman]{babel} % this is needed for umlauts
% \usepackage[T1]{fontenc}    % this is needed for correct output of umlauts in pdf
\usepackage[nonumber]{cuisine}
\usepackage{fontspec}

\setmainfont[Ligatures=TeX]{Comic Neue}
\setsansfont[Ligatures=TeX]{Comic Neue}

% \usepackage[margin=0.0cm, layoutwidth=6in, layoutheight=4in]{geometry}
\usepackage[a6paper, landscape, margin=0.0cm]{geometry}
% Total | number of servings | step number | ingredient | quantity | units
\RecipeWidths{146mm}{10em}{1em}{5em}{3em}{2em}
\renewcommand*{\recipetitlefont}{\large\bfseries}
\renewcommand*{\recipestepnumberfont}{\bfseries}
\renewcommand*{\recipefreeformfont}{\slshape}

\begin{document}

\begin{recipe}{Island Glazed Pork Tenderloin}{30 minutes}{}
  Preheat Oven to 350.
  \Ing{2 TBSP Olive Oil}
  Drizzle cast iron skillet with oil and preheat on stove till hot.
  
  \Ing{2 tsp salt}
  \Ing{1 tsp cinnamon}
  \Ing{1 tsp cumin}
  \Ing{1 tsp chili powder}
  \Ing{\fr12 tsp black pepper}
  Whisk together rub stuff
  \Ing{1 \fr12 lb Pork Tenderloin}
  Sprinkle rub stuff on tenderloin, massaging into the meat
  \newstep Brown tenderloin on skillet on all sides \textbf{3-4 min}
  \Ing{\fr34 cup brown sugar}
  \Ing{2 tsp minced garlic}
  \Ing{1 TBSP sriracha}
  Mix glaze and spread over tenderloin
  \newstep Transfer to oven, bake for \textbf{20 min} until cooked through
  \newstep Cut to \textbf{1 in} thick slices. Drizzle glaze from pan on loin before serving

  \end{recipe}

\end{document}