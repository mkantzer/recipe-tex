% !TEX TS-program = xelatex

% \documentclass[a6paper]{mikescards}
\documentclass[]{mikescards}

\usepackage{blindtext}
\usepackage{parcolumns}
% \usepackage{minipage}
% \usepackage{enumitem}

\begin{document}
\begin{recipe}{Japanese Milk Bread}{20 Min bake, 2.5hr prep}{Breadit}
% Source: https://old.reddit.com/r/Breadit/comments/v7q8iy/my_favorite_bread_recipe_of_all_time/ibn7dqu/
\begin{parcolumns}[colwidths={1=100pt}, rulebetween]{2}
  \colchunk{
    \begin{enumerate}
      \item \textbf{Tangzhong:}
      \item 20g Flour
      \item 27g Water
      \item 60g Milk
      \item 
      \item \textbf{Yeast Mix:} 
      \item 10g active dry yeast
      \item 40g lukewarm water
      \item 
      \item \textbf{Dough:}
      \item Tangzhong
      \item Yeast Mix
      \item 380g AP Flour
      \item 60g sugar
      \item 3g salt
      \item 130g milk
      \item 1 large egg
      \item 3 TBSP soft butter

      % \item 5-6 cloves garlic, minced
    \end{enumerate}
  }
  \colchunk{
    \begin{enumerate}

      \item Mix yeast and allow to bloom, \textbf{10 minutes}.
      \item 
      \item In a saucepan, add the tangzhong ingredients and cook, stirring over low until  paste forms. Should be thick and you will be able to pick all of it up with a spoon.
      \item 
      \item Add mixtures to remaining dough ingredients (NOT BUTTER) in a stand mixer bowl and mix on low speed for 5 minutes or until a proper dough forms.
      \item 
      \item Add butter 1 TBSP at a time. Mix on medium speed until fully incorporated and the dough is very smooth.
      \item 
      \item Cover and Proof in covered bowl for \textbf{1.5h} or until doubled in size.
      \item 
      \item Portion into 75g pieces, place into 9 inch cake pan w/ 1cm of space. Rest another \textbf{45 m}.
      \item 
      \item Brush with beaten egg, and bake \textbf{350F for 20 minutes}.
      \item 
      \item Top with garlic butter and enjoy.
    \end{enumerate}
  }
\end{parcolumns}



\end{recipe}
\end{document}
