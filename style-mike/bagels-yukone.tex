% !TEX TS-program = xelatex

% \documentclass[a6paper]{mikescards}
\documentclass[]{mikescards}

\usepackage{blindtext}
\usepackage{parcolumns}

\usepackage{fontspec}
\setmainfont[Ligatures=TeX]{Comic Neue}
\setsansfont[Ligatures=TeX]{Comic Neue}

\begin{document}
\begin{recipe}{Bagels}{60min or 28hr}{Serious Eats}
\begin{parcolumns}[colwidths={1=120pt}, rulebetween]{2}
  \colchunk{
    \begin{enumerate}
      \item \textbf{Yukone}
      \item 170g cold water
      \item 100g bread flour
      \item 
      \item \textbf{Dough}
      \item 355g bread flour
      \item 15g sugar
      \item 9g kosher salt
      \item 4g SAF instant dry yeast
      \item 100g water
      \item 
      \item \textbf{TO Boil}
      \item 30g barley malt syrup
      \item paper towels for draining
    \end{enumerate}
  }
  \colchunk{
    \begin{enumerate}
      \item \textbf{Yukone}
      \item In skillet, whisk together over medium heat. 
      \item \hspace{5pt} Go till they're like mashed potatoes, \textbf{2min}.
      \item Scrape onto plate, thinly. Cover and cool till \textbf{75\textdegree F}
      \item \textbf{Dough}
      \item Pulse flour, sugar, salt, yeast in a processor till combined. 
      \item Add cool yukone, water. Process till silky smooth, \textbf{~90sec}.
      \item \textbf{Shape}
      \item Turn onto un-flowered surface, and divide into 8 pieces \textbf{85g each}
      \item Cup a piece beneath palm. Quick, circular motions to make tight ball with tiny seam on bottom.
      \item Let rest \textbf{15min}.
      \item Poke hole in each with damp thumb, and stretch to ring.
      \item Place on greased pan, covered. Fridge \textbf{24-36h}.
      \item \textbf{Boil}
      \item Preheat oven \textbf{425\textdegree F}. Fill pot with 3 inch water, syrup. Boil.  
      \item 2-3 at a time, boil bagels, \textbf{30s} per side.
      \item Drain on paper towels \textbf{2-3s}, then transfer to a pan.
      \item \textbf{Bake: 25min, till blistered and golden brown}
      \item Cool at least \textbf{15min}
    \end{enumerate}
  }
\end{parcolumns}

\end{recipe}
\end{document}
