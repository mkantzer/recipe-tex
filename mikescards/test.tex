% !TEX TS-program = xelatex

% \documentclass[a6paper]{mikescards}
\documentclass[]{mikescards}

\usepackage{blindtext}
\usepackage{parcolumns}
% \usepackage{minipage}
% \usepackage{enumitem}

\begin{document}
\begin{recipe}{No-Knead Crusty White Bread}{40 Min bake, 5.55hr prep}{King Arthur Baking}

\begin{parcolumns}[colwidths={1=80pt}, rulebetween]{2}
  \colchunk{
    \begin{enumerate}
      \item 900g AP Flour
      \item 680g water, lukewarm
      \item 18g salt
      \item 14g instant yeast
    \end{enumerate}
  }
  \colchunk{
    \begin{enumerate}
      \item Combine all of the ingredients in a large mixing bowl
      \item Mix and stir everything together to make a very sticky, rough dough. If you have a stand mixer, beat at medium speed with the beater blade for 30 to 60 seconds.
      \item Let the dough rise, covered, at room temperature for 2 hours.
      \item Refrigerate for at least 2 hours, up to about 7 days. The longer you keep it in the fridge, the tangier it'll get.
      \item When you're ready to make bread, sprinkle the top of the dough with flour. Grease your hands, and pull off about a third of the dough.
      \item Plop the dough onto a floured work surface, and round it into a ball, or a longer log.
      \item Place on parchment-lined sheet. Sift a light coating of flour over the top. Drape the bread with greased plastic wrap, or cover it with a reusable cover.
      \item Let the loaf warm to room temperature and rise; 60 minutes. Preheat your oven to 450\textdegree F while the loaf rests. Place a shallow metal or cast iron pan on lowest oven rack, and have 1 cup of hot water ready to go.
      \item Make slices on top of bread
      \item Place the bread in the oven and carefully pour the 1 cup hot water into the pan. Close door quickly.
      \item Bake the bread for 25 to 35 minutes, until it's a deep, golden brown.
      \item Remove the bread from the oven, and cool it on a rack.
      \item Store leftover bread in a plastic bag at room temperature.
    \end{enumerate}
  }
\end{parcolumns}



\end{recipe}
\end{document}
